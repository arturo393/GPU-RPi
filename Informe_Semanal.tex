%%%%%%%%%%%%%%%%%%%%%%%%%%%%%%%%%%%%%%%%%%%%%%%%%%%%%%%%%%%%%%%%%%%%%%%%%%%
%
% Plantilla para un artculo en LaTeX en espaol.
%
%%%%%%%%%%%%%%%%%%%%%%%%%%%%%%%%%%%%%%%%%%%%%%%%%%%%%%%%%%%%%%%%%%%%%%%%%%%

\documentclass[11pt,oneside,titlepage]{article}

% Esto es para poder escribir acentos directamente:
%\usepackage[latin1]{inputenc}
\usepackage[utf8]{inputenc}
\usepackage{makeidx}
\usepackage{multirow}
\usepackage{titlesec}
\usepackage{sectsty}
\usepackage{fncychap}
\usepackage{color}
\usepackage{comment}

% Esto es para que el LaTeX sepa que el texto est en espaol:
\usepackage[spanish]{babel}
\usepackage[right=3cm,left=3cm,top=2.5cm,bottom=2.5cm,headsep=1cm,footskip=2cm]{geometry}
\usepackage{graphicx}
% Paquetes de la AMS:
\usepackage{amsmath, amsthm, amsfonts}
\usepackage{fancyhdr}
\pagestyle{fancy}
\lhead{
\chead{
\rhead{\bfseries Informe Semanal de Actividades Realizadas }
\lfoot{From: K. Grant}
\cfoot{To: Dean A. Smith}
\rfoot{\thepage}
\renewcommand{\headrulewidth}{0.4pt}
\renewcommand{\footrulewidth}{0.4pt}
}}

%\pagestyle{headings}
%\pagestyle{myheadings}
%\markright{Informe Semanal de Actividades Realizadas}
\begin{document}
\title{Informe Semanal de Actvividades Realizadas}
\author{Nombre: Arturo Veras\\ 
	Supervisor: Claudio Torres\\
	Empresa: CCTVAL \\
Tipo de Práctica: Profesional}
%\date{\color{green}December 2005}
%\maketitle
%\setlength{\unitlength}{1 cm} %Especificar unidad de trabajo
\thispagestyle{empty}
\begin{picture}(0,1.5)
\put(0,0){\includegraphics[width=2.7cm,height=2cm]{utfsm.jpg}}
\put(13,0){\includegraphics[width=2cm,height=2cm]{elo.jpg}}
\end{picture}
\\
\\
\begin{center}
\textbf{{\LARGE Universidad Técnica Federico Santa María}\\[0.5cm]
{\LARGE Departamento de Electrónica}}\\[4.25cm]
{\Large Informe de Práctica}\\[2.3cm]
{\LARGE \textbf{Centro Cient\'ifico Tecnol\'ogico de Valpara\'iso}}\\[3.5cm]
{\large Arturo Veras Olivos}\\[2cm]
Valparaiso - \today
\\
 {\large Versión 2.5}
\end{center}

%\newpage
%\tableofcontents
%\listoffigures % to produce list of figures
%\listoftables % to produce list of tables
%\newpage
\section*{Informe Semanal de Actividades Realizadas}
\begin{center}

\begin{tabular}{|r|l|}
\hline 
Nombre:  & Arturo Veras\\ 
\hline 
Supervisor & Claudio Torres \\ 
\hline 
Empresa: & CCTVAL \\ 
\hline 
Tipo de Práctica: & Profesional \\ 
\hline 
\end{tabular} 
\end{center}
\subsection*{Semana 1, del \textit{13/01/14 al 17/01/14}}
\begin{comment}
Tercera persona Él, Ella, Ello (del latín ille, illa, illud)
Se redacta en lenguaje formal y atemporal (no usar formas verbales en pasado simple)
Compró el periódico y se tomó un café.
\end{comment}

La primera tarea para esta semana es investigar formas de conectar una GPU (GT640 NVIDIA Graphic Card Unit) a una Raspberry Pi (Single Board Computer). Hay que tener encuentra que la interfaz de la tarjeta gr\'afica es PCIe 3.0 16x y la Raspberry Pi tiene el puerto GPIO (General-Purpose Input/Output) y USB 2.0 disponibles. Por lo tanto la primera estrategia es investigar si existe alg\'un dispositivo que haga de adaptador entre PCIe 16x y USB 2.0. 

Se propone conectar la GT640 a una tarjeta de desarrollo llamada \textbf{startKIT}, esta tarjeta posee una interfaz PCIe 1x y un puerto GPIO que se conecta facilmente a la Raspberry Pi. La idea es descartada porque no existen drivers para conectar una tarjeta de video, menos una NVIDIA, a este dispositivo. Además hay que desarrollar el driver para concetar la startKIT a la Raspberri Pi.

Otra idea que se propone es estudiar el funcionamiento de los adaptadores de PCIe a USB, particularmente el funcionamiento del chip el chip \textbf{MCS9901-CC} ya que, eventualmente, se puede diseñar una placa que hace la interfaz entre PCIe y USB en el modo que queremos.

La tercera idea que se propone es utilziar el adaptador de PCIe a Mini PCIe \textbf{PE4L-PM060}. La idea de este adaptador es utilizarlo en caso de que la Raspberry Pi no cuente con los requerimientos mínimos para conectarse a la GT640.

\subsection*{Semana 2, del 20/01/14 al 24/01/14}

Las tareas de esta semana consisten en seguir buscando otras alternativas ya que las
anteriores han sido descartas por el momento.

Se encuentra el chip \textbf{USB2380}, un controlador
PCIe 1.0 a USB 2.0. El problema es que para nuestro propósito, conectar la GT640 a través del puerto USB, se necesita
desarrollar un controlador propio. Por el momento creemos que esta solución esta
fuera de los objetivos principales, es más, aún no se posee información más
detallada sobre el desarrollo de dicha tarjeta. Otro problema que es que no
se sabe si podemos adquirir el dispositivo a tiempo para realizar pruebas ya que la compra se realiza en el extrangero.

Hasta la fecha no se ha encontrado un adaptador directo entre PCIe y USB, por lo tanto se decide por
estudiar la posibilidad de realizar una tarjeta que tenga ambas interfaces y que
cumpla con nuestros requerimientos. Las posibilidades son: \textbf{MCS990} PCIe
to 4-Port USB 2.0 Host Controller, este queda fuera porque no es posible realizar el adaptador,  y \textbf{USB2380} PCIe 1.0 to USB 2.0 controller aún se necesita buscar más información de como es el desarrollo del software controlador.

Miestras se contacta al desarrollador del chip USB2380 para más detalles se buscan otras alternavis para reemplazar la Raspberri Pi y que posean un slot PCIe. Se encuentra la tarjeta \textbf{SBC-A510 Single Board Computer}, esta tarjeta tiene un puerto mPCIe que podemos utilizar junto al adaptador \textbf{PE4L-PM060A}. Esta idea queda pendiente ya que la tarjeta es elevada y la idea principal es construir un dispositivo de bajo costo.

Finalmente para esta semana se concluye que no es posible la conexión de una tarjeta PCIe a la Raspberry Pi porque no cumple los requerimientos de hardware. A partir de la próxima semana la nueva estrategia será encontrar formas de que la GT640 actúe como un periférico USB con una conexión hot-plug.

\subsection*{Semana 3, del 27/01/14 al 31/01/14}

Una de las nuevas estrategias es utilizar una placa madre que tenga un puerto
PCIe. La idea es usar un hardware de bajo consto con los requerimientos mínimos
para que la GT640 funcione. Para esto existen alternativas como placas madres
Micro-ATX, Mini-ITX, Nano-ITX y Pico-ITX, claro está que a medido que la placa
es más pequeña más costosa será. 

Además, insvestigando hemos encontrado una solución muy parecida a lo que
queremos, se trata de una tarjeta desarrollo de llamada Kayla, básicamente es
un computador pequeño que reune todas las características para trabajar con la
tarjeta GT640.  

Se encontro otro dispositivo que en primera instancia cumple con nuestros
requerimientos. La \textbf{USB3380EVB} es una tarjeta de desarrollo que tiene la
version USB 3.0 del \textbf{USB2380} chip previamente mencionado. Esta tarjeta
posee un slot PCIe 1x. Envíamos correos preguntando la factibilidad de nuestro
proyecto.

Se encontro otra tarjeta de desarrollo, el \textbf{86 Duino plataform} la cual
posee el puerto mPCIe el cual podemos utilizar con el adaptador
\textbf{PE4L-PM060A}.

Hasta el momento tenemos las alternativas mencionada, realizamos el presupuesto y
estamos a la espera de la desición.

La alternativa más factible es comprar la \textbf{USB3380EV} y desarrollar el
software necesario para nuestra aplicación. El problema es que el distribuidor
esta de vacaciones por lo que tenemos que esperar hasta la próxima semana.

Mientras estudiamos la posibilidad de utilizar un computador de bajo costo como
interfaz entre la GPU y cualquier host. La conexión sería a trav\'es del USB de
forma plg and play.

La idea de conectar una RPi a la GPU se descarto completamente ya que la RPi no 
tiene la arquitectura necesaria para que CUDA, el software de desarrollo de las GPU de Nvidia, funcione.

\subsection*{Semana 4, del 03/02/14 al 06/02/14}

Esta semana se realizó una revisión de todas las ideas que estaban en el
tapete. Nos dimos cuenta que estabamos dando vueltas y era necesario tener una
nueva estrategia para resolver el problema. Decidimos volver a comenzar de
nuevo, ya con las demás ideas descartadas. Ya que aún no hemos tenido
respuestas satisfactorias sobre el controlador del USB3380 EVK, hemos decidido
construir un computador barato que cumpla con los requerimientos para instalar
la GT640 y CUDA. La idea sobre esto es tener un dispositivo portatil plug and
play, para esto estudiaremos la posibilidad de realizar a traves del puerto usb
o del ethernet. 

Por otro lado en caso de que compremos la tarjeta USB3380 EVK y el desarrollo
del driver sea ráido, hemos buscado computadores de bajo costo para realizar un
cluster. Hemos encontrado el ODROID. Es un computador muy barato para las
grandes características que posee, es compatublible con el driver NVidia y
CUDA.

\subsection*{Semana 5, del 10/02/14 al 14/02/14}
\begin{comment}
lunes
- Ademas se realizó un estudio sobre como conectar el comptuador de una forma
plug and play, las opciones son utilizar el puerto Ethernet o, mejor aún,
realizar la conexión a trav\'es USB.

martes 
- Reunion con el profesor para ver el estado del proyecto. 3 ideas
  nuevas aparecieron.

miercoles 
- Contacto con un desarrollador de USB3380 para solicitar ayuda en el
  campo. 
- Estudio de RNDIS para conectar usb a usb, existe la posibilidad.

jueves 
- Estudio del funcionamiento de usb. Aprendi que podemos utilizar el
  computador como gadget y realizar la configuración recompilando el kernel y
  agregando los drivers.Mas detalles leer Linux Gadget Drivers
- Aún falta el cable. 
- Compilación del kernel para agregar los módulos. Estamos a la espera
  del cable USB macho macho.

viernes 
- Update del proyecto al profesor Claudio torres. 
 - compramos cable USB 3.0, estamos a la espera del envío.
\end{comment}


\subsection*{Semana 6, del 17/02/14 al 21/02/14}
\begin{comment}
lunes 
- Se adquirio el cable, se connecto pero no funciono. Se revisa si los
módulos estan compilados.  
-  Estudiar la posibilidad de crear un driver.  
- Se manda correo a lista-usb. Espera de respuesta.

martes -  No es posible conectar el cable usb entre PC. El hardware USB que
traen los PCI no es soportado por el driver linux gadget que es el encargado de
realizar esta conexión, 
- Hemos descartado la posibilidad de conectar PC to PC
con un cable simple.  Decidimos realizar la conexión a través de ethernet.

miercoles 
- NAAAAAAAAAAAAADAAAAAAAAAAAAAAA 

jueves 
- NNNNNNNNNNNNAAAAAAAAAAAADAAAAAAAaaaa 
viernes 
- NAAAAAAAAAAAAAAAADAAAAAAAAAAAa

\end{comment}

\subsection*{Semana 6, del 24/02/14 al 28/02/14}
\begin{comment}

lunes
- Se decide por comprar la tarjeta desarrollo USB33880 , Claudio Torres no
escucha nuestras advertencias de que hay que realiar el driver completamente y
aún asi decide comprarla. Francisco nos pide que le mandemos el valor para que
la Universidad realice la compra, el producto no es disponible. Mantemos al
tanto a Torres sobre el problema, no sabemos que hacer.  Por mientras cada uno
hace lo suyo.
-  

lunes
\end{comment}

\end{document}

